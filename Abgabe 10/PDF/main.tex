% <page>
    \documentclass[12pt, a4paper]{article}
    \usepackage[margin=1cm,headheight=1cm]{geometry}
    \geometry{
        a4paper,
        total={210mm,297mm},
        left=20mm,
        right=20mm,
        top=35mm,
        bottom=30mm,
    }
% </page>

% <font>
	% http://tex.stackexchange.com/questions/59702/suggest-a-nice-font-family-for-my-basic-latex-template-text-and-math
    % 	sans-serif-font: helvet
    \usepackage{kpfonts} 
    % 	sans-serif --> \sfdefault
    \renewcommand{\familydefault}{\rmdefault}
    \usepackage{inconsolata}
    \usepackage[utf8]{inputenc}
    \usepackage[german]{babel}
    \usepackage[document]{ragged2e}
    \usepackage[babel, german=quotes]{csquotes}
    \usepackage{textcomp}
    \usepackage{enumitem}
    \usepackage{perpage}
    \MakePerPage{footnote}
    \usepackage{xcolor}
    \usepackage{wasysym} 
% </font>

% <header_footer>
	\usepackage{fancyhdr}
    \pagestyle{fancy}
    \fancyhf{}
    \renewcommand{\headrulewidth}{0pt}
    
    \usepackage{lastpage}
    \rfoot{Seite {\thepage} von \pageref{LastPage}}

	% <document>
      \rhead{bearbeitet von Gabriel Engel, \\ Robert Kobiella, \\ Michael Neu\hphantom{,}}
      \lhead{\textbf{Abgabe 10} für \\ \textit{Algorithmen und Datenstrukturen} bei \\ Prof. Dr. Carsten Kern}
      \title{Algorithmen und Datenstrukturen - Abgabe 10}
	% </document>

% </header_footer>

% <code>
    \usepackage{minted}
    \renewcommand{\theFancyVerbLine}{\sffamily \textcolor[rgb]{0.3,0.3,0.3}{\scriptsize \oldstylenums{\arabic{FancyVerbLine}}}}
    \usemintedstyle{vs}
% </code>
 
% <math>
    \usepackage{amssymb}
    \usepackage{amsmath}
    \usepackage{eqnarray}
    \usepackage{stmaryrd}

    \newcommand{\nonterm}[1]{\langle #1 \rangle}
    \newcommand{\nuffsaid}{\hfill $\Box$}
% </math>


% <tikz>
    \usepackage{tikz}
    \usetikzlibrary{automata, positioning}
	\usetikzlibrary{decorations.pathreplacing}

    \tikzset{
        clr_red/.style = {red},
        clr_blue/.style = {blue},
        clr_green/.style = {green}
    }
    \tikzset{
        between/.style args={#1 and #2}{
             at = ($(#1)!0.5!(#2)$)
        }
    }
    
	\usepackage{forest}
    \forestset{qtree edges/.style={for tree={parent anchor=south, child anchor=north}}}
% </tikz>

% <tables>
    \usepackage{tabularx}
    \usepackage{longtable}
	\renewcommand{\arraystretch}{1.2}
% </tables>


\begin{document}
	%% standard-tabelle
\begin{tabularx}{\textwidth}{|X|X|X|X|}
	\hline
    content & in & 4 & spalten \\
    \hline
\end{tabularx}

% tabelle über mehrere seiten (header kommt automatisch)
% 3x 2cm-spalten
\begin{longtable}{*{3}{|p{2cm}}|}
	\hline
	\textbf{Titel} & \textbf{der} & \textbf{Spalten} \\
    \hline
    \endhead
    normaler & content & hier \\
    \hline
\end{longtable}

% alphabetische aufzählung
\begin{enumerate}[nolistsep, noitemsep, label=\alph*)]
	\item content
    \item content
\end{enumerate}

% code
\begin{minted}[mathescape=false, fontsize=\fontsize{9.5pt}{10.8pt}, xleftmargin=6mm, framesep=0mm, tabsize=4, linenos]{python}
import antigravity
\end{minted}

% eigene schriftgröße                      {  x  }{ 1.2*x }
\newcommand{\schriftgroessenname}{\fontsize{8.5pt}{10.2pt}}

% lange gleichungen (aligned)
\begin{align*}
	a &= b \\
    &= \nonterm{f} \lightning \\
%
% fallunterscheidung
	fib(n) &= \begin{cases}
                \quad fib(n - 1) + fib(n - 2) & \text{, falls $n > 2$} \\
                \quad 1 & \text{, sonst}
            \end{cases}
\end{align*}

% beweis ende
\nuffsaid

% baum
\begin{center}
	\begin{forest}
        qtree edges
        [$A$, name=NODE_A
        	[$B$ 
            	[$D$ ] 
                [$E$ ] 
			]
            [$C$, name=NODE_C
            	[$F$ ]
                [,.phantom]
			]
		]
%
		\draw[decorate,decoration={brace,amplitude=10pt}] (3,0) -- (3,-2.5);
        \node[anchor=west] at (3.5,-1.3) {Beschreibung};
%
		\draw[dashed, <->] (NODE_C) to [bend right=90] (NODE_A);
	\end{forest}
\end{center}

% graphen
\begin{center}
	\begin{tikzpicture}[node distance=3cm]
    	\node[graph node] (v1) {$v_1$};
        \node[graph node, above right=1.5cm and 2cm of v1] (v2) {$v_2$};
        \node[graph node, below right=1.5cm and 2cm of v1] (v3) {$v_3$};


		\draw[graph edge] (v1) -- (v2) node[edge weight] {12};
		\path[graph edge] (v1)  edge[bend left=20] node[edge weight] {2} (v3)
        		          (v3) edge[bend left=20] node[edge weight] {3} (v1);

		\draw[graph edge] (v2) -- (v3) node[edge weight] {1};
    \end{tikzpicture}
\end{center}

% typographie
\textbf{fett}
\textit{kursiv}
\underline{unterstrichen}
\texttt{monospace/code}












% font test
Neutra photo booth yuccie roof party, williamsburg sartorial art party iPhone sustainable fixie kogi. Bushwick kinfolk authentic, roof party blue bottle meditation banh mi cliche skateboard echo park. Artisan gastropub kombucha tilde retro sartorial. Cronut whatever brunch umami. Banjo butcher chartreuse marfa. Hella pabst iPhone shabby chic brunch. YOLO gentrify meditation, skateboard direct trade roof party butcher tumblr narwhal heirloom. Sustainable vinyl vice, whatever franzen pork belly actually pour-over gluten-free. Microdosing sartorial cray paleo next level messenger bag.


\newpage
	\section*{Aufgabe 1}
\paragraph{Idee} \enquote{Erweiterte Tiefensuche}: Soll ein bereits besuchter Knoten besucht werden, so wird dieser Zyklus ausgegeben. 

\paragraph{Implementierung}
\begin{minted}[mathescape=false, fontsize=\fontsize{9.5pt}{10.8pt}, xleftmargin=6mm, framesep=0mm, tabsize=4, linenos]{csharp}
public List<GraphNode> GetCycle()
{
	var cycle = new List<GraphNode>();

	foreach (var node in nodes)
	{
		cycle = GetCycle(node, new List<GraphNode>());

		if (cycle.Count > 0)
		{
			break;
		}
	}

	return cycle;
}

private List<GraphNode> GetCycle(GraphNode start, List<GraphNode> currentPath)
{
	foreach (var node in GetConnectedNodes(start))
	{
		if (currentPath.Contains(node))
		{
			return  currentPath.SkipWhile(x => x != node)
								.Add(node)
								.ToList();
		}
		else
		{
			var tempPath = new List<GraphNode>();
			tempPath.AddRange(currentPath);
			tempPath.Add(node);

			var cycle = GetCycle(node, tempPath);

			if (cycle.Count > 0)
			{
				return cycle;
			}
		}
	}

	return new List<GraphNode>();
}
\end{minted}
	\section*{Aufgabe 2}

\begin{minipage}{0.5\textwidth}
    \begin{minted}[mathescape=false, fontsize=\fontsize{9.5pt}{10.8pt}, xleftmargin=6mm, framesep=0mm, tabsize=4, linenos]{python}
def mod_1(n): 
	return n - int(n)

def h(s): 
	m = 1000
	x = (5**0.5 - 1) / 2

	return int(m * mod_1(s * x))

for i in [61, 62, 63, 64, 65]: 
	print i, h(i)
    \end{minted}
\end{minipage}
\begin{minipage}{0.5\textwidth}
	\begin{align*}
    	h(61) &= 700 \\
        h(62) &= 318 \\
        h(63) &= 936 \\
        h(64) &= 554 \\
        h(65) &= 172
    \end{align*}
\end{minipage}
    \newpage
	\section*{Aufgabe 3}
\begin{minted}[mathescape=false, fontsize=\fontsize{9.5pt}{10.8pt}, xleftmargin=6mm, framesep=0mm, tabsize=4, linenos]{python}
#!/usr/bin/env python

import random
from math import floor

def countsort(lst, k):
	counter = [0] * (k + 1)

	for i in lst:
		counter[i] += 1
 
	ndx = 0;
	for i in range(len(counter)):
		while counter[i] > 0:
			lst[ndx] = i
			ndx += 1
			counter[i] -= 1

	return lst

def heapsort(lst):
	def buildMaxHeap(lst, start, end):
		root = start

		while True:
			child = root * 2 + 1

			if child > end:
				break

			if child + 1 <= end and lst[child] < lst[child + 1]:
				child += 1

			if lst[root] < lst[child]:
				lst[root], lst[child] = lst[child], lst[root]
				root = child
			else:
				break
	  
	# inline heapify
	for start in range((len(lst)-2)/2, -1, -1):
		buildMaxHeap(lst, start, len(lst)-1)
 
	for end in range(len(lst)-1, 0, -1):
		lst[end], lst[0] = lst[0], lst[end]
		buildMaxHeap(lst, 0, end - 1)

	return lst

def mapsort(lst, c):
	newn = int(len(lst) * c)
	bucket = [-1] * newn
	minVal, maxVal = min(lst), max(lst)

	dist = (maxVal - minVal) / float(newn - 1)

	for i in range(len(lst)):
		t = int(floor((lst[i]-minVal) / dist))
		insert = lst[i]
		left = 0

		if bucket[t] != -1 and insert <= bucket[t]:
			left = 1;

		while bucket[t] != -1:
			if left == 1:
				if insert > bucket[t]:
					insert, bucket[t] = bucket[t], insert
				if t > 0:
					t -= 1
				else:
					left = 0
			else:
				if insert <= bucket[t]:
					insert, bucket[t] = bucket[t], insert
				if t < newn - 1:
					t += 1
				else:
					left = 1

		bucket[t] = insert;

	j = 0
	for i in range(newn):
		if bucket[i] != -1:
			lst[j] = bucket[i]
			j += 1

	return lst

def generate_list():
	return [random.randint(1000, 10000) for r in range(random.randint(10, 1000))]
\end{minted}

zu b) Eine nicht repräsentative Messung bei 1000 Ausführungen pro Algorithmus hat im Schnitt folgende Ausführungszeiten ergeben:
\begin{itemize}[noitemsep]
  \item Countsort: $0,012 s$
  \item Heapsort: $0,014 s$
  \item Mapsort: $0,008 s$
\end{itemize}
    \newpage
	\section*{Aufgabe 4}
Implementierung erfolgte in Python, vgl. Anhang. 

\begin{minted}[mathescape=false, fontsize=\fontsize{9.5pt}{10.8pt}, xleftmargin=6mm, framesep=0mm, tabsize=4, linenos]{python}
from hash import *

def test_table(implementation, numbers): 
	collisions = 0

	for i in numbers: 
		collisions += implementation.insert(i)

	return (implementation.table, collisions)

if __name__ == "__main__": 
	size = 11
	numbers = [10, 22, 31, 4, 15, 28, 17, 88, 59]

	print "Linear:", test_table(LinearHashtable(size), numbers)
	print "Quadratic:", test_table(QuadraticHashtable(size), numbers)
	print "Double:", test_table(DoubleHashtable(size), numbers)
\end{minted}

\vspace{0.5cm}

\begin{tabularx}{\textwidth}{|X|l||*{11}{r|}}
	\hline
    \textbf{Sondierung} & \textbf{Kollisionen} & 0 & 1 & 2 & 3 & 4 & 5 & 6 & 7 & 8 & 9 & 10 \\
    \hline
    Linear & \hphantom{1}7 & 22 & 88 &  &  & 4 & 15 & 28 & 17 & 59 & 31 & 10 \\
\hline
Quadratisch & 14 & 22 &  & 88 & 17 & 4 &  & 28 & 59 & 15 & 31 & 10 \\
\hline
Doppeltes Hashing & \hphantom{1}7 & 22 &  & 59 & 17 & 4 & 15 & 28 & 88 &  & 31 & 10 \\
\hline
\end{tabularx}

\end{document}