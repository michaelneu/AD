\newpage

\section*{Aufgabe 4}
	a) Implementierung in C\# (Programmiersprache im Anhang kann sich unterscheiden)
	\begin{minted}[framesep=0mm, linenos, xleftmargin=6mm, fontsize=\footnotesize]{csharp}
public class Algorithm {
    public static int Recursion(int n, int m) {
        if (n == 0) {
            return m + 1;
        } else if (m == 0 && n >= 1) {
            return Recursion(n - 1, 1);
        } else {
            return Recursion(n - 1, Recursion(n, m - 1));
        }
    }

    public static int Iteration(int n, int m) {
        Stack<int> stack = new Stack<int>();
        
        stack.Push(n);
        stack.Push(m);

        while (stack.Count != 0) {
            m = stack.Pop();
            n = stack.Pop();

            if (n == 0) {
                // recursion end

                if (stack.Count > 0) {
                    // coming from previous "inner recursion"
                    stack.Pop();
                    n = stack.Pop();

                    // return from "inner recursion"
                    stack.Push(n);
                    stack.Push(m + 1);
                }
            } else if (m == 0 && n >= 1) {
                stack.Push(n - 1);
                stack.Push(1);
            } else {
                // outer recursion call
                stack.Push(n - 1);
                stack.Push(m);

                // inner recursion call
                stack.Push(n);
                stack.Push(m - 1);
            }
        }

        return m + 1;
    }
}
	\end{minted}
	\newpage
    
    b) Induktionsbeweis \\
    \begin{itemize}[noitemsep]
    	\item[I.A.1:] $f(0, m) = m + 1$ für ein beliebiges $m$
        \item[I.V.1:] $f(n, m)$ terminiert für ein festes aber beliebiges $n$
        \item[I.S.1:] Fallunterscheidung $f(n + 1, m)$
        	\subitem \begin{itemize}[noitemsep]
                \item[1. Fall] trifft nicht ein, da $n + 1 > 0$
                \item[2. Fall] $f(n + 1 - 1, 1) = f(n, 1)$ terminiert nach I.V.
                \item[3. Fall] $f(n + 1 - 1, f(n + 1, m - 1)) = f(n, f(n + 1, m - 1))$ terminiert, 
                \item[] falls $f(n + 1, m - 1)$ terminiert (für jedes $m$)
                \item[I.A.2:] $m = 1$: $f(n + 1, 1 - 1) = f(n, 0) = f(n - 1, 1)$ terminiert für jedes $n$ (siehe I.V.1)
                \item[I.V.2:] $f(n + 1, m - 1)$ terminiert für beliebiges aber festes $m$
                \item[I.S.2:] $m + 1$: $f(n + 1, m)$
                    \subitem \begin{itemize}[noitemsep]
                        \item[1. Fall] $n = 0$ tritt nicht ein, da $n + 1 > 0$
                        \item[2. Fall] $f(n, 1)$ terminiert nach I.V.
                        \item[3. Fall] $f(n, f(n + 1, m - 1))$ terminiert, da $f(n + 1, m - 1)$ terminiert nach I.V.2
                    \end{itemize}
            \end{itemize}
	\end{itemize}
    \nuffsaid \\[1cm]
    
    c) Folgende Tabelle wurde unter .NET 4.5 auf Windows 7 erzeugt. Beim Wertepaar $n = 10$, $m = 3$ tritt in der rekursiven Implementierung ein Stack Overflow auf.  \\[0.5cm]
    
    \bgroup
		\def\arraystretch{1.2}
        
        \begin{longtable}{|p{0.5cm}|p{0.5cm}|p{2cm}|p{2cm} || p{0.5cm}|p{0.5cm}|p{2cm}|p{2cm}|}
            \hline
            \textbf{m} & \textbf{n} & \textbf{Rekursion} & \textbf{Iteration} & \textbf{m} & \textbf{n} & \textbf{Rekursion} & \textbf{Iteration} \\
            \hline
    		\endhead
            0 & 0 & 1 & 1 & 0 & 1 & 2 & 2 \\
            \hline
            0 & 2 & 3 & 3 & 0 & 3 & 4 & 4 \\
            \hline
            0 & 4 & 5 & 5 & 0 & 5 & 6 & 6 \\
            \hline
            0 & 6 & 7 & 7 & 0 & 7 & 8 & 8 \\
            \hline
            0 & 8 & 9 & 9 & 0 & 9 & 10 & 10 \\
            \hline
            0 & 0 & 11 & 11 & 1 & 0 & 2 & 2 \\
            \hline
            1 & 1 & 3 & 3 & 1 & 2 & 4 & 4 \\
            \hline
            1 & 3 & 5 & 5 & 1 & 4 & 6 & 6 \\
            \hline
            1 & 5 & 7 & 7 & 1 & 6 & 8 & 8 \\
            \hline
            1 & 7 & 9 & 9 & 1 & 8 & 10 & 10 \\
            \hline
            1 & 9 & 11 & 11 & 1 & 0 & 12 & 12 \\
            \hline
            2 & 0 & 3 & 3 & 2 & 1 & 5 & 5 \\
            \hline
            2 & 2 & 7 & 7 & 2 & 3 & 9 & 9 \\
            \hline
            2 & 4 & 11 & 11 & 2 & 5 & 13 & 13 \\
            \hline
            2 & 6 & 15 & 15 & 2 & 7 & 17 & 17 \\
            \hline
            2 & 8 & 19 & 19 & 2 & 9 & 21 & 21 \\
            \hline
            2 & 0 & 23 & 23 & 3 & 0 & 5 & 5 \\
            \hline
            3 & 1 & 13 & 13 & 3 & 2 & 29 & 29 \\
            \hline
            3 & 3 & 61 & 61 & 3 & 4 & 125 & 125 \\
            \hline
            3 & 5 & 253 & 253 & 3 & 6 & 509 & 509 \\
            \hline
            3 & 7 & 1021 & 1021 & 3 & 8 & 2045 & 2045 \\
            \hline
            3 & 9 & 4093 & 4093 & \textbf{3} & \textbf{10} & \textbf{8189} & \textbf{8189} \\
            \hline
        \end{longtable}
    \egroup