\section*{Aufgabe 4}
Der Algorithmus für \textit{MaxTeilsumme4-2d} wurde im Folgenden in C++ umgesetzt, der mitunter verwendete Algorithmus für \textit{MaxTeilsumme3} entspricht dem der Vorlesung und besitzt die Laufzeit $O(n)$. Die Programmiersprache der Implementierung im Anhang kann sich unterscheiden. 

\begin{minted}[framesep=0mm, linenos, xleftmargin=6mm, fontsize=\fontsize{8.5pt}{10.2pt}, tabsize=4]{cpp}
#include <algorithm>
#include <iostream>
#include <limits>

int maxPartialSum(int a[], size_t size) {
	int i, s, curSum = 0,
		maxSum = std::numeric_limits<int>::min();

	for (int i = 0; i < size; ++i) {
		s = curSum + a[i];

		if (s > a[i])
			curSum = s;
		else
			curSum = a[i];

		maxSum = std::max(curSum, maxSum);
	}

	return maxSum;
}

template <size_t elements>
int maxPartialSum2D(int (&a)[elements][elements], size_t size) {
	int *temp = new int[size];
	int maxSum = std::numeric_limits<int>::min();

	for (int left = 0; left < size; ++left) {
		std::memset(temp, 0, size * sizeof(int));

		for (int right = left; right < size; ++right) {
			for (int i = 0; i < size; ++i)
				temp[i] += a[i][right];

			maxSum = std::max(maxPartialSum(temp, size), maxSum);
		}
	}
	
    delete[] temp;
	return maxSum;
}

int main(int argc, const char **argv) {
	const size_t n = 5;

	int matrix[n][n] = {
		{1, 2, 3, 4, 5}, 
        {-1, -2, -3, -5, -8}, 
        {-3, 1, 4, -1, 6}, 
        {2, 4, -9, 16, 25},
		{-100, 81, 64, -49, 36}
	};

	std::cout << "Maximum partial sum is " << maxPartialSum2D(matrix, n) << std::endl;
	return 0;
}

\end{minted}