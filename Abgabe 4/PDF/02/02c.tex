c) Selectionsort
\begin{minted}[framesep=0mm, linenos, xleftmargin=6mm, fontsize=\fontsize{8.5pt}{10.2pt}, tabsize=4]{csharp}
public void Sort(int[] array)
{
    for (int i = array.Length - 1; i >= 0; i--)
    {
        int max = i;

        for (int j = i - 1; j >= 0; j--)
        {
            if (array[j] > array[max])
            {
                max = j;
            }
        }

        int temp = array[i];
        array[i] = array[max];
        array[max] = temp;
    }
}
\end{minted}

\textbf{Korrektheit}

Behauptung: Nach jedem $i$-ten Schleifendurchlauf gilt (Schleifeninvariante): $a_{n-i} \leq \ldots \leq a_n$ mit $n = |array| - 1$
\begin{itemize}[noitemsep]
    \item[\textbf{I.A.:}] $i = 0$: 
    \item[] \begin{itemize}[nolistsep, noitemsep]
    	\item In der äußeren Schleife wird $i = |array| - 1$ und $max = i$
        \item In der inneren Schleife wird $j = i - 1 \not \geq 0$. Die Vorbedigung der Schleife ist nicht erfüllt, der Rumpf wird nicht ausgeführt.
        \item Vertauschung von $a[i]$ mit $a[max] = a[i]$. Es gilt $a_{n-i} = a_n \leq a_n$
    \end{itemize}
    \item[\textbf{I.V.:}] Behauptung gilt für $i -1$
    \item[\textbf{I.S.:}] $i-1 \rightarrow i$
    \item[] \begin{itemize}[nolistsep, noitemsep]
    	\item Nach I.V. gilt $a_{n-i} \leq a_{n-(i-1)} \leq \ldots \leq a_n$
        \item Im $i$-ten Schleifendurchlauf ist $i$ der Index des letzten Elements der unsortierten Folge
        \item In der inneren Schleife wird die Position des Maximums der unsortieren Folge bestimmt und in $max$ gespeichert. Pro Schleifendurchlauf wird der Laufindex um $1$ verringert und nicht ungültig.
        \item Das letzte Element der unsortierten Folge ($a[i]$) wird mit dem Maximum ebendieser ($a[max]$) vertauscht, so dass: $a_{n-i} \leq a[max] = a_{n-(i-1)} \leq \ldots \leq a_n$
        \item D.h. $a_{n-i} \leq \ldots \leq a_n$
    \end{itemize}
    \item[$\Rightarrow$] Behauptung gilt für alle $i \in \mathbb{N}$
\end{itemize} \nuffsaid

\textbf{Laufzeit}: $T(n) = \frac{n(n-1)}{2} \in \Theta(n^2)$