d) Quicksort
\begin{minted}[framesep=0mm, linenos, xleftmargin=6mm, fontsize=\fontsize{8.5pt}{10.2pt}, tabsize=4]{csharp}
private int PreparePartition(int[] array, int first, int last)
{
    int index = RandomIndex(first, last),
    Swap(array, index, first);

    int pivot = array[first],
        partition = first - 1;

    for (int i = first; i <= last; i++)
    {
        if (array[i] <= pivot)
        {
            partition++;

            Swap(array, i, partition);
        }
    }

    Swap(array, first, partition);

    return partition;
}

private void QuickSort(int[] array, int first, int last)
{
    if (first < last)
    {
        int partition = PreparePartition(array, first, last);

        QuickSort(array, first, partition - 1);
        QuickSort(array, partition + 1, last);
    }
}

public void Sort(int[] array)
{
    QuickSort(array, 0, array.Length - 1);
}
\end{minted}
\vspace{-0.3cm}
\textbf{Korrektheit} \\
\begin{itemize}[nolistsep, noitemsep]
    \item Aufrufbedingung für \texttt{PreparePartition}: $first, last \in \{0, \dots, |a| - 1\}$ und $first < last$
    \item $index \in \{first, \dots, last\}$ 
    \item Vertauschen von von $a[index]$ mit $a[first]$
    \item Pivot-Element ist das erste Element des Bereichs $a[first:last]$, somit das zuvor vertauschte zufällige Element
    \item $partition = first - 1$
    \item Beginn des \enquote{eigentlichen} Sortierens, Iteration über den Bereich $a[first:last]$
    \item[] \begin{itemize}[nolistsep, noitemsep]
    	\item $i \in \{first, last\}$, $i$ wird jeden Schleifendurchlauf um 1 erhöht
        \item Fallunterscheidung: 
        \item[] \begin{itemize}[nolistsep, noitemsep]
        	\item[1. Fall:] $a[i] \leq pivot$: $partition$ wird um 1 nach rechts verschoben, Element $a[i]$ wird mit dem Element links neben der Grenze vertauscht
            \item[] $\Rightarrow a[partition - 1] \leq a[i]$
            \item[2. Fall:] $a[i] > pivot$: Nichts wird verändert
        \end{itemize}
    \end{itemize}
    \item Vertauschen des Elements $a[first]$ mit $a[partition]$
    \item $\Rightarrow a[first] \leq \dots \leq a[partition] \leq \dots \leq a[last]$
\end{itemize}

\nuffsaid


\textbf{Laufzeit: } Da das Vertauschen des ersten Elements mit einem zufälligen Element des Bereiches $a[first:last]$ eine atomare Operation ist änder sich die Laufzeit des Algorithmus nicht und somit bleibt diese $\Theta(n \cdot \log(n))$