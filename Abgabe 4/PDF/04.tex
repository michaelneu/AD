\section*{Aufgabe 4}
a) Insertionsort (rekursiv)
\begin{minted}[framesep=0mm, linenos, xleftmargin=6mm, fontsize=\fontsize{8.5pt}{10.2pt}, tabsize=4]{csharp}
private void Sort(int[] array, int start)
{
    if (start < array.Length)
    {
        int key = array[start],
            i = start - 1;

        while (i >= 0 && array[i] > key)
        {
            array[i + 1] = array[i];
            i--;
        }

        array[i + 1] = key;
        Sort(array, start + 1);
    }
}

public void Sort(int[] array)
{
    Sort(array, 0);
}
\end{minted}


\textbf{Laufzeit/Rekursionsgleichung}
\begin{itemize}[noitemsep]
	\item Basis: $T(1) = 4$
	\item $T(n) = T(n-1) + 4n = T(n-2) + 4(n-1) + 4n = \ldots = \displaystyle \sum_{i=1}^{n} 4n = \frac{4n(n-1)}{2}$
	\item[$\Rightarrow$] $T(n) \in \Theta(n^2)$
\end{itemize}


b) Mergesort\footnote{Die Methode \texttt{Merge} entspricht der aus den Vorlesungsunterlagen} (iterativ)
\begin{minted}[framesep=0mm, linenos, xleftmargin=6mm, fontsize=\fontsize{8.5pt}{10.2pt}, tabsize=4]{csharp}
public void Sort(int[] array)
{
    for (int i = 1; i < array.Length; i *= 2)
    {
        for (int j = 0; j < array.Length; j += 2 * i)
        {
            int first = j,
                last = Math.Min(j + i, array.Length - 1);
            
            Merge(array, first, last, (first + last) / 2);
        }
    }
}
\end{minted}

\textbf{Laufzeit: } Die Halbierungen der Felder benötigen insgesamt  $\Theta(\log(n))$ Zeitschritte. Das Mischen (in der Funktion \textit{Merge}) benötigt pro Iteration weitere $\Theta(n)$ Zeitschritte. \\
$\Rightarrow T(n) \in \Theta(n \log(n))$ (selbe Laufzeit wie rekursive Implementierung)
