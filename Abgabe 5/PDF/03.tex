\section*{Aufgabe 3}
a) Aus Vorlesung bekannt: $n$ Elemente sind in $i$ Schichten \enquote{untergebracht}. 
\begin{align*}
	\sum_{k=0}^i \left( 2^k \right) &= \frac{2^{i + 1} - 1}{2 - 1} \\
    &= 2^{i + 1} - 1 &\text{(Elemente im Baum)} \\[0.5cm]
    2^{i + 1} - 1 &= n \\
    2^{i + 1} &= n + 1 \\
    i + 1 &= \log(n + 1) \\
    i &\geq \log(n) \\
    i &= \lfloor \log(n) \rfloor
\end{align*}

b) Behauptung: $n_h = \lceil \frac{n}{2^{h+1}} \rceil$
\begin{itemize}[noitemsep]
	\item[\textbf{I.A.:}] $h = 0;$
    $n_0 = \lceil \frac{n}{2^{0 + 1}} \rceil$ \\
    für $n=7 \Rightarrow n_0 = 4$ $\checkmark$
	\item[\textbf{I.V.:}] Behauptung gilt für ein $n \in \mathbb{N}$
	\item[\textbf{I.S.:}]
      \begin{align*}
      	n_{h+1} &= \left\lceil \frac{n}{2^{(h + 1) + 1}} \right\rceil \\
        &= \left\lceil \frac{1}{2} \cdot \frac{n}{2^{h+1}} \right\rceil \\
        &= \left\lceil \frac{1}{2} \cdot n_h \right\rceil
      \end{align*}    
    \item[$\Rightarrow$] Nach dem Prinzip der vollständigen Induktion gilt die Behauptung für alle $n \in \mathbb{N}$
\end{itemize}

c) Für alle $x$ mit $|x| < 1$ gilt: 
\begin{align*}
	\sum_{k=0}^\infty x^k &= \frac{1}{1 - x} &&\text{geometrische Reihe} \\
    \sum_{k=0}^\infty \left( k \cdot x^{k - 1} \right) &= \frac{1}{(1 - x)^2} &&\text{beide Seiten differenzieren} \\
    \sum_{k=0}^\infty \left( k \cdot x^k \right) &= \frac{x}{(1 - x)^2} &&\text{beide Seiten mit $x$ multiplizieren}
\end{align*}

d) Die Reihenfolge kann vertauscht werden, da die Methode \texttt{Heapify} per Rekursion sicherstellt, dass auch alle Unterbäume einem MaxHeap entsprechen. 