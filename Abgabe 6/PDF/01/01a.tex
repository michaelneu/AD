a) Quicksort auf einer verketteten Liste \\
\begin{minted}[mathescape=false, fontsize=\fontsize{9.5pt}{10.8pt}, xleftmargin=6mm, framesep=0mm, tabsize=4, linenos]{csharp}
private ListElement<T> PreparePartition(ListElement<T> first, ListElement<T> last)
{
    T pivot = first.Data;
    var partition = first.Previous;
    var element = first;

    while (element != null && element != last.Next)
    {
        if (element.Data.CompareTo(pivot) <= 0)
        {
            if (partition == null)
            {
                partition = first;
            }
            else
            {
                partition = partition.Next;
            }

            Swap(element, partition);
        }

        element = element.Next;
    }

    Swap(first, partition);

    return partition;
}

private void Quicksort(ListElement<T> first, ListElement<T> last)
{
    if (first != null && last != null && first != last && first.Previous != last)
    {
        var partition = PreparePartition(first, last);

        if (partition != null)
        {
            Quicksort(first, partition.Previous);
            Quicksort(partition.Next, last);
        }
    }
}
\end{minted}

\vspace{0.5cm}

\paragraph{Speicherplatz} Misst man mit den C\#-Garbage-Collection-Tools den Speicherverbrauch mittels \texttt{GC.GetTotalMemory(true)}, so ergibt sich folgende grobe Abschätzung des Platzbedarfes einer Liste der Länge $n$:
\begin{align*}
    T_{Memory}(n) &= (n + 1)\cdot 20 + \text{sizeof(System.Console)} + 4 \cdot (n + 1) \\
    [T_{Memory}] &= \text{Byte}
\end{align*}

\paragraph{Laufzeit} Die hier gezeigte Implementierung entspricht der Implementierung des Quicksorts auf einem Array, beide finden In-Place auf dem Array/der Liste statt und werden über die selben rekursiven Bedingungen definiert. Die Methode \texttt{PreparePartition} iteriert über die Elemente von \texttt{first} bis \texttt{last}, exakt wie die Array-Implementierung. Da die Laufzeit dieser Implementierung bekannt ist, entspricht die Quicksort-Implementierung auf einer verketteten Liste dieser, weshalb sie sich in $\Theta(n \cdot \log(n))$ befindet. \\