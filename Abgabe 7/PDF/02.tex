\section*{Aufgabe 2}
a) Minimaler AVL-Baum der Höhe 5 \\
\begin{center}
	\begin{forest}
        qtree edges
        [{}, draw , circle [{}, draw , circle [{}, draw , circle [{}, draw , circle [{}, draw , circle [{}, draw , circle [,.phantom] [,.phantom] ] [,.phantom] ] [{}, draw , circle [,.phantom] [,.phantom] ] ] [{}, draw , circle [{}, draw , circle [,.phantom] [,.phantom] ] [,.phantom] ] ] [{}, draw , circle [{}, draw , circle [{}, draw , circle [,.phantom] [,.phantom] ] [,.phantom] ] [{}, draw , circle [,.phantom] [,.phantom] ] ] ] [{}, draw , circle [{}, draw , circle [{}, draw , circle [{}, draw , circle [,.phantom] [,.phantom] ] [,.phantom] ] [{}, draw , circle [,.phantom] [,.phantom] ] ] [{}, draw , circle [{}, draw , circle [,.phantom] [,.phantom] ] [,.phantom] ] ] ]
	\end{forest}
\end{center}

b) Um alle möglichen minimalen AVL-Bäume generieren zu können benötigt man die Eigenschaft, dass die Höhe des linken und rechten Teilbaums um maximal 1 unterscheiden darf. Eine mögliche Implementierung befindet sich im Anhang. \\[0.5cm]

c) Es gibt 4096 verschiedene AVL-Bäume der Höhe 5. 