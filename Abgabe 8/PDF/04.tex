\section*{Aufgabe 4}
\paragraph{Entstehende Bäume} Der Einfachheit halber (sowohl aus Copy-Paste-Sicht, als auch aus Compile-Zeit-Sicht) werden hier nur exemplarisch einige entstehende Bäume dargestellt, das Vorgehen zum Löschen wurde analog ausgeführt. \\[0.5cm]

\paragraph{Löschen von 26}
\begin{center}
	\begin{forest}
    	qtree edges
        [13, draw 
        	[{3, 6, 10}, draw 
            	[{1, 2}, draw] 
                [{4, 5}, draw] 
                [{7, 8, 9}, draw] 
                [{11, 12}, draw]
            ] 
            [{18, 21}, draw 
            	[{14, 15, 16, 17}, draw] 
                [{19, 20}, draw] 
                [{22, 23, 24, 25}, draw]
            ]
        ]
    \end{forest}
\end{center}

\paragraph{Löschen von 25, 24, 23, 22, 21}
\begin{center}
	\begin{forest}
    	qtree edges
        [13, draw 
        	[{3, 6, 10}, draw 
            	[{1, 2}, draw] 
                [{4, 5}, draw] 
                [{7, 8, 9}, draw] 
                [{11, 12}, draw]
            ] 
            [{18}, draw 
            	[{14, 15, 16, 17}, draw] 
                [{19, 20}, draw]
            ]
        ]
    \end{forest}
\end{center}

\paragraph{Löschen von 20}
\begin{center}
	\begin{forest}
    	qtree edges
        [10, draw 
        	[{3, 6}, draw 
            	[{1, 2}, draw] 
                [{4, 5}, draw] 
                [{7, 8, 9}, draw] 
            ] 
            [{13, 17}, draw 
                [{11, 12}, draw]
            	[{14, 15, 16}, draw] 
                [{18, 19}, draw]
            ]
        ]
    \end{forest}
\end{center}

\paragraph{Löschen von 2}
\begin{center}
	\begin{forest}
    	qtree edges
        [1, draw]
    \end{forest}
\end{center}

\paragraph{Löschen von 1}
\begin{center}
	\begin{forest}
    	qtree edges
        [null, draw, dashed]
    \end{forest}
\end{center}