% <page>
    \documentclass[12pt, a4paper]{article}
    \usepackage[margin=1cm,headheight=1cm]{geometry}
    \geometry{
        a4paper,
        total={210mm,297mm},
        left=20mm,
        right=20mm,
        top=35mm,
        bottom=30mm,
    }
% </page>

% <font>
	% http://tex.stackexchange.com/questions/59702/suggest-a-nice-font-family-for-my-basic-latex-template-text-and-math
    % 	sans-serif-font: helvet
    \usepackage{kpfonts} 
    % 	sans-serif --> \sfdefault
    \renewcommand{\familydefault}{\rmdefault}
    \usepackage{inconsolata}
    \usepackage[utf8]{inputenc}
    \usepackage[german]{babel}
    \usepackage[document]{ragged2e}
    \usepackage[babel, german=quotes]{csquotes}
    \usepackage{textcomp}
    \usepackage{enumitem}
    \usepackage{perpage}
    \MakePerPage{footnote}
    \usepackage{xcolor}
    \usepackage{wasysym}
% </font>

% <header_footer>
	\usepackage{fancyhdr}
    \pagestyle{fancy}
    \fancyhf{}
    \renewcommand{\headrulewidth}{0pt}
    
    \usepackage{lastpage}
    \rfoot{Seite {\thepage} von \pageref{LastPage}}

	% <document>
      \rhead{bearbeitet von Gabriel Engel, \\ Robert Kobiella, \\ Michael Neu\hphantom{,}}
      \lhead{\textbf{Abgabe 8} für \\ \textit{Algorithmen und Datenstrukturen} bei \\ Prof. Dr. Carsten Kern}
      \title{Algorithmen und Datenstrukturen - Abgabe 8}
	% </document>

% </header_footer>

% <code>
    \usepackage{minted}
    \renewcommand{\theFancyVerbLine}{\sffamily \textcolor[rgb]{0.3,0.3,0.3}{\scriptsize \oldstylenums{\arabic{FancyVerbLine}}}}
    \usemintedstyle{vs}
% </code>
 
% <math>
    \usepackage{amssymb}
    \usepackage{amsmath}
    \usepackage{eqnarray}
    \usepackage{stmaryrd}

    \newcommand{\nonterm}[1]{\langle #1 \rangle}
    \newcommand{\nuffsaid}{\hfill $\Box$}
% </math>


% <tikz>
    \usepackage{tikz}
    \usetikzlibrary{automata, positioning}
	\usetikzlibrary{decorations.pathreplacing}

    \tikzset{
        clr_red/.style = {red},
        clr_blue/.style = {blue},
        clr_green/.style = {green}
    }
    \tikzset{
        between/.style args={#1 and #2}{
             at = ($(#1)!0.5!(#2)$)
        }
    }
    
	\usepackage{forest}
    \forestset{qtree edges/.style={for tree={parent anchor=south, child anchor=north}}}
% </tikz>

% <tables>
    \usepackage{tabularx}
    \usepackage{longtable}
	\renewcommand{\arraystretch}{1.2}
% </tables>


\begin{document}
	%% standard-tabelle
\begin{tabularx}{\textwidth}{|X|X|X|X|}
	\hline
    content & in & 4 & spalten \\
    \hline
\end{tabularx}

% tabelle über mehrere seiten (header kommt automatisch)
% 3x 2cm-spalten
\begin{longtable}{*{3}{|p{2cm}}|}
	\hline
	\textbf{Titel} & \textbf{der} & \textbf{Spalten} \\
    \hline
    \endhead
    normaler & content & hier \\
    \hline
\end{longtable}

% alphabetische aufzählung
\begin{enumerate}[nolistsep, noitemsep, label=\alph*)]
	\item content
    \item content
\end{enumerate}

% code
\begin{minted}[mathescape=false, fontsize=\fontsize{9.5pt}{10.8pt}, xleftmargin=6mm, framesep=0mm, tabsize=4, linenos]{python}
import antigravity
\end{minted}

% eigene schriftgröße                      {  x  }{ 1.2*x }
\newcommand{\schriftgroessenname}{\fontsize{8.5pt}{10.2pt}}

% lange gleichungen (aligned)
\begin{align*}
	a &= b \\
    &= \nonterm{f} \lightning
\end{align*}

% beweis ende
\nuffsaid

% baum
\begin{center}
	\begin{forest}
        qtree edges
        [$A$, name=NODE_A
        	[$B$ 
            	[$D$ ] 
                [$E$ ] 
			]
            [$C$, name=NODE_C
            	[$F$ ]
                [,.phantom]
			]
		]
%
		\draw[decorate,decoration={brace,amplitude=10pt}] (3,0) -- (3,-2.5);
        \node[anchor=west] at (3.5,-1.3) {Beschreibung};
%
		\draw[dashed, <->] (NODE_C) to [bend right=90] (NODE_A);
	\end{forest}
\end{center}

% typographie
\textbf{fett}
\textit{kursiv}
\underline{unterstrichen}
\texttt{monospace/code}














% font test
Neutra photo booth yuccie roof party, williamsburg sartorial art party iPhone sustainable fixie kogi. Bushwick kinfolk authentic, roof party blue bottle meditation banh mi cliche skateboard echo park. Artisan gastropub kombucha tilde retro sartorial. Cronut whatever brunch umami. Banjo butcher chartreuse marfa. Hella pabst iPhone shabby chic brunch. YOLO gentrify meditation, skateboard direct trade roof party butcher tumblr narwhal heirloom. Sustainable vinyl vice, whatever franzen pork belly actually pour-over gluten-free. Microdosing sartorial cray paleo next level messenger bag. \\ Disrupt lomo mustache, flexitarian messenger bag keytar schlitz. Waistcoat wayfarers bitters pabst, etsy vinyl franzen mixtape ethical cray thundercats skateboard authentic knausgaard. Viral taxidermy artisan raw denim lumbersexual, small batch trust fund salvia kinfolk irony XOXO tattooed umami. Thundercats hoodie +1 salvia intelligentsia banh mi. Tacos mlkshk everyday carry fap selvage. Letterpress seitan kogi lo-fi marfa. Church-key raw denim stumptown, shabby chic flannel tote bag normcore. Master cleanse tacos humblebrag green juice, taxidermy tousled twee portland. Paleo williamsburg portland helvetica celiac. Disrupt freegan bitters, readymade literally green juice meditation gentrify single-origin coffee godard hella pour-over tattooed listicle. Kitsch hammock disrupt, drinking vinegar paleo tattooed gentrify intelligentsia chartreuse flexitarian. Cronut mumblecore try-hard, fingerstache typewriter before they sold out gentrify waistcoat pug. \\ Thundercats actually occupy chia. Lomo DIY post-ironic VHS next level offal. 8-bit yr four dollar toast bespoke godard, gastropub portland trust fund paleo intelligentsia. Jean shorts craft beer before they sold out farm-to-table, etsy austin kogi iPhone. Put a bird on it farm-to-table plaid migas, freegan letterpress DIY gluten-free church-key microdosing brunch. Chia kogi PBR\&B occupy drinking vinegar shoreditch, swag literally. Raw denim microdosing offal, jean shorts scenester williamsburg helvetica +1 aesthetic etsy leggings narwhal.
	\section*{Aufgabe 1}
\begin{minipage}{0.5\textwidth}
    \begin{minted}[mathescape=false, fontsize=\fontsize{9.5pt}{10.8pt}, xleftmargin=6mm, framesep=0mm, tabsize=4, linenos]{python}
def h(s): 
	return s % 9

for i in [5, 28, 19, 15, 20, 33, 12, 17, 10]: 
	print i, h(i)
    \end{minted}
\end{minipage}
\begin{minipage}{0.5\textwidth}
	\begin{tabularx}{\textwidth}{|*{10}{X|}}
		\hline
		0 & 1 &  2 &  3 & 4 & 5 &  6 & 7 &  8 \\
		\hline
		  & 28 & 20 & 12 &   & 5 & 15 &   & 17 \\
		  & 19 &    &    &   &   & 33 &   &    \\
		  & 10 &    &    &   &   &    &   &    \\
		\hline
	\end{tabularx}
\end{minipage}

    \newpage
	\section*{Aufgabe 2}

\begin{minipage}{0.5\textwidth}
    \begin{minted}[mathescape=false, fontsize=\fontsize{9.5pt}{10.8pt}, xleftmargin=6mm, framesep=0mm, tabsize=4, linenos]{python}
def mod_1(n): 
	return n - int(n)

def h(s): 
	m = 1000
	x = (5**0.5 - 1) / 2

	return int(m * mod_1(s * x))

for i in [61, 62, 63, 64, 65]: 
	print i, h(i)
    \end{minted}
\end{minipage}
\begin{minipage}{0.5\textwidth}
	\begin{align*}
    	h(61) &= 700 \\
        h(62) &= 318 \\
        h(63) &= 936 \\
        h(64) &= 554 \\
        h(65) &= 172
    \end{align*}
\end{minipage}
    \newpage
	\section*{Aufgabe 3}
\begin{minted}[mathescape=false, fontsize=\fontsize{9.5pt}{10.8pt}, xleftmargin=6mm, framesep=0mm, tabsize=4, linenos]{csharp}
public static bool NaiveSearch(string text, string muster)
{
	int i = 0;
	
	while (text[i] != muster[0] && i < text.Length)
	{
		i++;
	}

	for (int j = 0; j < muster.Length; j++)
	{
    	if (i + j == text.Length - 1 || text[i + j] != muster[j])
    	{
    		return false;
    	}
	}

	return true;
}
\end{minted}

\paragraph{Laufzeit} NaiveSearch \\
\begin{itemize}[nolistsep, noitemsep]
    \item \texttt{while}-Schleife iteriert $n$ mal, oder bricht vorher ab, falls erstes Zeichen übereinstimmt
    \item bei Abbruch iteriert \texttt{for}-Schleife bis maximal $n$ weiter
\end{itemize}

\textbf{somit gilt:} $\Theta(n)$.
    \newpage
	\section*{Aufgabe 4}
	Sei $M$ eine Matrix der Größe $n \times n$, dann benötigt der Additionsalgorithmus für $M + M$ genau $n \cdot n = n^2$ Einzelschritte, der Multiplikationsalgorithmus für $M \cdot M$ benötigt $n \cdot n \cdot n = n^3$ Einzelschritte zur Berechnung des Ergebnisses. Testet man nun mit einer bspw. $500 \times 500$ Matrix wie viele Einzelschritte die CPU im Durchschnitt pro Millisekunde durchführen kann, so kann man dieses Ergebnis skalieren und durch Ziehen der 2. bzw. 3. Wurzel aus dem Ergebnis die maximale Größe der Matrix berechnen. \\
    Folgende Tabelle wurde auf einem Intel Core i7-2670QM auf Windows 7 unter .NET 4.5 erzeugt: \\[0.5cm]
    
    \bgroup
	\def\arraystretch{1.2}
        \begin{tabularx}{\textwidth}{|l|X|X|}
            \hline
            \textbf{Ziel} & \textbf{Größe der Matrix für Addition} & \textbf{Größe der Matrix für Multiplikation} \\
            \hline
            1 Minute & 44424 $\times$ 44424 & 1241 $\times$ 1241 \\
            \hline
            2 Minuten & 62826 $\times$ 62826 & 1563 $\times$ 1563 \\
            \hline
            5 Minuten & 99337 $\times$ 99337 & 2122 $\times$ 2122 \\
            \hline
            10 Minuten & 140483 $\times$ 140483 & 2674 $\times$ 2674 \\
            \hline
        \end{tabularx}
    \egroup
\end{document}